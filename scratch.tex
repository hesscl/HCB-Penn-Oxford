
    \only<2>{\item These trends together suggest racial segregation is becoming more suburban, a trend which may require us to rethink how we }
    \only<2>{\item Housing availability and affordability are theoretically important factors in this process that may drive differentiation between suburban locations.}

%literature on segregation
\begin{frame}{Literature Review}
\begin{itemize}
    \item Massey and Denton
    \item Sharkey / Patillo McCoy
    \item Lichter, Parisi, Tacquino articles
\end{itemize}
\end{frame}

%literature about metropolitan inversion / suburbanization / suburbanization of poverty 
\begin{frame}{Literature Review}
\begin{itemize}
    \item Suburbanization of Poverty
    \item ``Demographic Inversion''
\end{itemize}
\end{frame}

%what questions does the review of literature raise and why are they important?
\begin{frame}{Research Questions \& Hypotheses}
    \begin{enumerate}
        \item To what extent are the typical neighborhood conditions in urban, older suburban and newer suburban neighborhoods associated with metropolitan residential segregation, and have these contrasts changed over time?
        \begin{itemize}
            \item Hypothesis 1:
        \end{itemize}
        \item What are the micro-level dynamics that explain aggregate racial differences in exposure to high-poverty neighborhoods across urban and suburban locations?
        \begin{itemize}
            \item Hypothesis 2:
        \end{itemize}
    \end{enumerate}
\end{frame}